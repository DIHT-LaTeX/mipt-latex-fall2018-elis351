\documentclass[12pt]{article}

\usepackage[T2A]{fontenc}
\usepackage[utf8]{inputenc}
\usepackage[english,russian]{babel}

\title{Домашняя работа №1}
\author{Елизавета Пузына}
\date{}

\begin{document}
	\maketitle
	\begin{flushright}

	\textit{Audi multa,}
	
	\textit{loquere pauca}

	\end{flushright}

	\begin{flushleft}
	{\vspace{20pt} \hspace{14pt}}Это первый документ в системе компьютерной вёрстки {\LaTeX}.
	\end{flushleft}
	
	\begin{center}
	{\huge \textsf{<<Ура!!!>>}}
	\end{center}
	
	\begin{flushleft}
	{\hspace{14pt}}А теперь формулы. {\textsc{Формула}} - краткое и точное словесное выражение, определение или же  ряд математических величин, выраженный условными знаками.
	
    {\vspace{15pt} \hspace{28pt} \bfseries \Large Термодинамика}
	
	{\hspace{14pt}}Уравнение Менделеева-Кланейрона --- уравнение состояния идеального газа, имеющее вид ${pV = vRT}$, где ${p}$ --- давление, ${V}$ --- объем, занимаемый газом, ${T}$ --- температура газа, ${v}$ --- количество вещества газа, а ${R}$ --- универсальная газовая постоянная.
	
	{\vspace{15pt} \hspace{28pt} \bfseries \Large Геометрия \hfill \bfseries \Large Планиметрия}
	
	{\hspace{14pt}}Для плоского треугольника со сторонами ${a, b, c}$ и углом ${\alpha}$, лежащим против стороны ${a}$, справедливо соотношение
	$$
	a^2 = b^2 + c^2 - 2bc {\cos{\alpha}},
	$$
	из которого можно выразить косинус угла треугольника:
	$$
	{\cos{\alpha}} = \frac{b^2 + c^2 - a^2}{2bc},
	$$
	
	{\hspace{14pt}}Пусть ${p}$ --- полупериметр треугольника, тогда путем несложных преобразований можно получить, что
	$$
	{\tg {\frac{\alpha}{2}} = {\sqrt{\frac{(p-b)(p-c)}{p(p-a)}}}}
	$$
	{\vspace{1cm}}На сегодня, пожалуй, хватит\dots Удачи!
	\end{flushleft}
\end{document}
