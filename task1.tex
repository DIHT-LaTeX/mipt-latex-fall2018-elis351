\documentclass[12pt]{article}

\usepackage[T2A]{fontenc}
\usepackage[utf8]{inputenc}
\usepackage[russian]{babel}

\title{Домашняя работа №1}
\author{Елизавета Пузына}
\date{}

\begin{document}
	\maketitle
	\begin{flushright}
	\textit{Audi multa,\\ loquere pauca}
	\end{flushright}

	\vspace{20pt}Это первый документ в системе компьютерной вёрстки \LaTeX.
	
	\begin{center}
	\huge \textsf{<<Ура!!!>>}
	\end{center}
	
	А теперь формулы. \textsc{Формула}~--- краткое и точное словесное выражение, определение или же ряд математических величин, выраженный условными знаками.
	
    \vspace{15pt} \hspace{14pt} {\bfseries \Large Термодинамика}
	
	Уравнение Менделеева--Кланейрона~--- уравнение состояния идеального газа, имеющее вид $pV = \nu RT$, где $p$~--- давление, $V$~--- объем, занимаемый газом, $T$~--- температура газа, $\nu$~--- количество вещества газа, а $R$~--- универсальная газовая постоянная.
	
	{\vspace{15pt} \hspace{14pt} \bfseries \Large Геометрия \hfill Планиметрия}
	
    Для плоского треугольника со сторонами $a$, $b$, $c$ и углом $\alpha$, лежащим против стороны $a$, справедливо соотношение
	$$
	a^2 = b^2 + c^2 - 2bc \cos \alpha,
	$$
	из которого можно выразить косинус угла треугольника:
	$$
	\cos \alpha = \frac{b^2 + c^2 - a^2}{2bc},
	$$
	
	Пусть $p$~--- полупериметр треугольника, тогда путем несложных преобразований можно получить, что
	$$
	\tg \frac{\alpha}{2} = \sqrt{\frac{(p-b)(p-c)}{p(p-a)}}
	$$ \\[1cm]
	На сегодня, пожалуй, хватит\dots Удачи!
\end{document}
